\documentclass{article}
\usepackage{authblk}
\usepackage{amsmath}
\usepackage{amssymb}
\usepackage{comment}
\usepackage{hyperref}
\usepackage{longtable}
\usepackage{stmaryrd}
\newcommand{\interp}[1]{\llbracket #1 \rrbracket}
\newcommand{\maps}{\colon}
\renewcommand{\:}{\colon}
\newcommand{\FinSet}{\mathrm{FinSet}}
\newcommand{\Set}{\mathrm{Set}}
\newcommand{\Cat}{\mathrm{Cat}}
\newcommand{\Calc}{\mathrm{Calc}}
\newcommand{\Mon}{\mathrm{Mon}}
\newcommand{\BoolAlg}{\mathrm{BoolAlg}}
\renewcommand{\Form}{\mathrm{Form}}
\newcommand{\leftu}{\mathrm{left}}
\newcommand{\rightu}{\mathrm{right}}
\newcommand{\send}{\mathrm{send}}
\newcommand{\recv}{\mathrm{recv}}
\newcommand{\comm}{\mathrm{comm}}
\renewcommand{\quote}[1]{``#1"}
\newcommand{\deref}[1]{\mathrm{eval}(#1)}
\newcommand{\op}{\mathrm{op}}
\newcommand{\NN}{\mathbb{N}}
\newcommand{\pic}{$\pi$-calculus}

\title{Logic as a distributive law}
\author[1]{Michael Stay \thanks{\fontsize{8}{8}\selectfont stay@pyrofex.net}}
\author[2]{L.G. Meredith \thanks{\fontsize{8}{8}\selectfont president@rchain.coop}}
\affil[1]{Pyrofex Corp.}
\affil[2]{RChain Coop.}
\begin{document}
\maketitle

\begin{abstract}
  \noindent
  We introduce an algorithm for generating a type system from three
  data: a notion of computation, a notion of collection, and a
  ``distributive law'' between them. In the simplest form of the
  algorithm the notions of computation and collection are captured by
  Lawvere theories, respectively, and the distributive law is between
  the monads corresponding to each Lawvere theory. We look at various
  enrichments of the notions of collection and computation to extend
  the scope of the algorithm and also compare it to an alternative
  approach to generating a logic, namely the internal logic of a
  category. Among the many benefits of the algorithm we find a
  unification of several core logical concepts, most strikingly we
  find that the standard function type from the typed lambda calculus
  can be unified with the modalities from Hennessy-Milner logic.
\end{abstract}

\section{Introduction and motivation}
\subsection{What is the logic as a distributive law research
  programme? (LGM)}
By 2005 a number of research efforts made it clear that many different
biological systems, from cell signaling regimes to large organism
immune responses, could be effectively modeled in mobile process
calculi such as the \pic and the ambient
calculus. \cite{CardelliBio}\cite{Priami}\cite{Aviv} One of
the present authors imagined a future in which there were several
large scale databases capturing the dynamics of many different
biological regimes as \pic terms and wondered what sort of
query language might be used to search such a database. Following the
insights of relational databases, which build on relational algebras,
the question can be framed as

\[\mathsf{SQL} : \mathsf{relational algebra} :: \mathsf{X} : \mathsf{a process calculus}\]

Intriguingly, Caires' semantics for his spatial-behavioral
logic for the \pic has exactly the shape required of a
semantics of a query language. Putting formulae in the role of
queries, the semantics supplies just those terms that satisfy the
query.

One fundamental challenge with this approach, however, at least as
regards the biological domain, is that there is no-one-size-fits-all
model, i.e. different biological subdomains are best served by
different process calculi. \cite{CardelliSurvey}. Looking at the range
of Hennessy-Milner style logics, from the original proposals to the
spatial-behavioral systems one sees that these are largely constructed
from the same basic elements. Taking a clue from Abramksy's Domain
Theory in Logical Form, it seems reasonable to ask whether there is a
functorial characterization of these different logics, so that given a
model of computation, in this case the operational semantics of a
process calculus, and some other ingredients one could generate a
logic the semantics of which would match a suitable query model.

Note that were such a characterization to be found the scope of
applications is very broad. In particular, biology is far from the
only source of repositories of behavioral models. Code is the dark
matter of the Internet. Whether we are talking about large scale open
source repositories, such as github, or code repositories behind the
firewalls of the enterprise, code is one of the Internet's largest
data assets. Yet, it is not treated as a data asset in the sense that
it is not considered searchable in the same way data is
searchable. This represents a significant opportunity.

Note also that this view of the desiderata lines up with key
developments in logic. Specifically, the query semantics suggested here picks out
just those computations, such as \pic processes, from a given source
of computations, such as a repository or database, that bear witness to a particular
property. In other words, this approach to the design of a query model
shares much with the realizability approach to semantics of logic. 
\subsection{Propositions/filters as types, structural type systems, realizability (LGM)}
\subsection{Generation of type systems for programming languages (Javascript, Python, et al) (LGM)}
  Sums of languages (MAS)
\subsection{Type systems as search mechanisms (LGM)}
  Service discovery and type-directed service synthesis (LGM)
\subsection{Related work}
  Refinement types (MAS)

  - Inheritance synchronization (LGM)

  Abramsky's domain theory in logical form (LGM)

  Topos theory is special case of LADL where collection monad is covariant powerset monad. (MAS)
\subsection{Outline of paper}

\section{Finite structures as an example (MAS)}
Collection calculus C

Comprehension calculus R

Term calculus T

(C+R)+T -> (C+R)(T) ?

\section{What are the limits of the notion of collection?}
Motivation: Scala repeatedly has revised its collection library repeatedly and the community agrees that it still isn't right.  Having a proper theory of collection would help with the design of the library.

Criteria
\begin{itemize}
  \item inhabitation/addressibility (location within collection, more refined than inhabitation) 
  \item we can take the derivative of Cont, so we can talk about locations in any monad, not just finitary ones.
  \item For finitely presentable semantic collections the same thing works
  \item composition (lists of lists)
  \item decomposition
  \item subcollections
  \item Only the free algebra
\end{itemize}

Binary trees as unital magmas: pairing, empty tree (leaves are nodes with both children empty)

Ternary trees aren't magmas, but we still think of them as a collection.

Do we allow arbitrary algebraic data types?  What does it mean for one of those to be empty?
The development of a comprehension type. (LGM + MAS)

\section{Conclusions and future work (LGM + MAS)}
Arrow as modality

How lack of powerset, universal quantification, top affects operational semantics.

%\bibliographystyle{amsplain}
%\bibliography{ladl}
\end{document}
