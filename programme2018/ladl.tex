\documentclass{article}
\usepackage{amsmath}
\usepackage{amssymb}
\usepackage{comment}
\usepackage{hyperref}
\usepackage{longtable}
\usepackage{stmaryrd}
\newcommand{\interp}[1]{\llbracket #1 \rrbracket}
\newcommand{\maps}{\colon}
\renewcommand{\:}{\colon}
\newcommand{\FinSet}{\mathrm{FinSet}}
\newcommand{\Set}{\mathrm{Set}}
\newcommand{\Cat}{\mathrm{Cat}}
\newcommand{\Calc}{\mathrm{Calc}}
\newcommand{\Mon}{\mathrm{Mon}}
\newcommand{\BoolAlg}{\mathrm{BoolAlg}}
\renewcommand{\Form}{\mathrm{Form}}
\newcommand{\leftu}{\mathrm{left}}
\newcommand{\rightu}{\mathrm{right}}
\newcommand{\send}{\mathrm{send}}
\newcommand{\recv}{\mathrm{recv}}
\newcommand{\comm}{\mathrm{comm}}
\renewcommand{\quote}[1]{``#1"}
\newcommand{\deref}[1]{\mathrm{eval}(#1)}
\newcommand{\op}{\mathrm{op}}
\newcommand{\NN}{\mathbb{N}}

\title{Logic as a distributive law}
\author{
Michael Stay\inst{1}\\
\and
L.G. Meredith\inst{2}\\
}
\institute{
  {Pyrofex Corp.}\\
  \email{\fontsize{8}{8}\selectfont stay@pyrofex.net}\\
  \and
  {RChain Coop.}\\
  \email{\fontsize{8}{8}\selectfont president@rchain.coop}
}
\begin{document}
\maketitle

\begin{abstract}
\noindent
\end{abstract}

\section{Introduction and motivation}
\subsection{What is the logic as a distributive law research programme? (LGM)}
\subsection{Propositions/filters as types, structural type systems, realizability (LGM)}
\subsection{Generation of type systems for programming languages (Javascript, Python, et al) (LGM)}
  Sums of languages (MAS)
\subsection{Type systems as search mechanisms (LGM)}
  Service discovery and type-directed service synthesis (LGM)
\subsection{Related work}
  Refinement types (MAS)

  - Inheritance synchronization (LGM)

  Abramsky's domain theory in logical form (LGM)

  Topos theory is special case of LADL where collection monad is covariant powerset monad. (MAS)
\subsection{Outline of paper}

\section{Finite structures as an example (MAS)}
Collection calculus C

Comprehension calculus R

Term calculus T

(C+R)+T -> (C+R)(T) ?

\section{What are the limits of the notion of collection?}
Motivation: Scala repeatedly has revised its collection library repeatedly and the community agrees that it still isn't right.  Having a proper theory of collection would help with the design of the library.

Criteria
\begin{itemize}
  \item inhabitation/addressibility (location within collection, more refined than inhabitation) 
  \item we can take the derivative of Cont, so we can talk about locations in any monad, not just finitary ones.
  \item For finitely presentable semantic collections the same thing works
  \item composition (lists of lists)
  \item decomposition
  \item subcollections
  \item Only the free algebra
\end{itemize}

Binary trees as unital magmas: pairing, empty tree (leaves are nodes with both children empty)

Ternary trees aren't magmas, but we still think of them as a collection.

Do we allow arbitrary algebraic data types?  What does it mean for one of those to be empty?
The development of a comprehension type. (LGM + MAS)

\section{Conclusions and future work (LGM + MAS)}
Arrow as modality

How lack of powerset, universal quantification, top affects operational semantics.

%\bibliographystyle{amsplain}
%\bibliography{ladl}
\end{document}
